\section{Conclusioni}
A fronte dello studio riportato in questo documento risulta evidente come i casi di attacchi(tra Following Algorithm e Following Car) alla posizione e quelli all'accelerazione (con valore positivo) possano essere identificati come i casi più critici in quanto portano con estrema probabilità ad un incidente tra i veicoli. \\
Ad opinione degli autori di questo documento sarebbe opportuno investire risorse per contrastare queste casistiche rendendo il sistema più tollerante: ad esempio aggiungere ridondanza tra i collegamenti per individuare condizioni di attacco. \\
Attacchi all'accelerazione con valore pari a 0 risultano scaturire in comportamenti variabili a seconda del tempo di attacco.\\
Attacchi all'accelerazione con valori negativi risultano invece meno critici dal punto di vista degli incidenti, i quali risultano essere altamente improbabili nel dominio della simulazione.