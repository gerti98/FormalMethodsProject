\section{Scelte di Sviluppo}
\subsection{Modifica FMU}
Gli attacchi verrano implementati utilizzando la tecnica del \textit{Man-in-the-Middle}: verrà introdotta una FMU semplificata tra un punto di comunicazione di due FMU, questo consentirà di semplificare la modifica dell'implementazione dell'attacco in quanto non è necessario conoscere i dettagli implementativi delle FMU in gioco. Questo a patto di un maggior overhead del sistema per effettuare la comunicazione dei parametri tra le varie FMU.
\subsection{Scelta dei parametri}

\begin{itemize}
	\item \textbf{Step-size}: \textbf{0.01s}. E' un buon tradeoff tra un sensoring più preciso ed una durata di simulazione accetabile.
	\item \textbf{Tempo di Simulazione}: \textbf{100s}
\end{itemize}